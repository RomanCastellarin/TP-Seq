\documentclass[12pt]{article}
\usepackage[utf8]{inputenc}
\usepackage[spanish]{babel}
\usepackage[pdftex]{graphicx}
\usepackage{float}
\usepackage{booktabs}
\usepackage[table,xcdraw]{xcolor}
\usepackage{ltxtable}
\usepackage{listings}
\lstset{language=Haskell}
\usepackage{color}
\begin{document}
\begin{titlepage}

\begin{minipage}{2.6cm}
\includegraphics[width=\textwidth]{fceia.pdf}
\end{minipage}
\hfill
%
\begin{minipage}{6cm}
\begin{center}
\normalsize{Universidad Nacional de Rosario\\
Facultad de Ciencias Exactas,\\
Ingeniería y Agrimensura\\}
\end{center}
\end{minipage}
\hspace{0.5cm}
\hfill
\begin{minipage}{2.6cm}
\includegraphics[width=\textwidth]{unr.pdf}
\end{minipage}

\vspace{0.5cm}

\begin{center}
\normalsize{\sc Estructuras de Datos II}\\
\vspace{0.5cm}
\large{Trabajo Práctico II}\\

\Large{\bf Secuencias}\\
\vspace{5cm}

\normalsize
Román Castellarin\\
Juan Ignacio Suarez\\

\vspace*{0.5cm}
\small{ \today }


\end{center}
\end{titlepage}
\newpage
\section{Introducción}

\subsection*{Implementacion basada en Arrays Persistentes}

\begin{tabular}{@{}lcc@{}}
\toprule
        & W & S \\ \midrule
filterS & $O(N)$  &  $O(N)$  \\
reduceS & $O(N)$  &  $O(N)$ \\
scanS   & $O(N)$  &  $O(N)$ \\
showtS  & $O(N)$  &  $O(N)$ \\ \bottomrule
\end{tabular}

\subsubsection{El trabajo de filterS es $O(N)$}

\begin{table}[h]
\begin{lstlisting}
filterA p x = A.flatten (mapA g x)
    where g y = if p y then singletonA y else emptyA
\end{lstlisting}
\caption{Definicion de filterA}
\end{table}
\begin{itemize}
\item \textbf{Lema:} \texttt{singletonA} y \texttt{emptyA} son $\Theta(1)$ en trabajo y profundidad.\\
 \textit{Dem:} Ambas hacen una cantidad constante de operaciones independientemente de la entrada.
 
\item \textbf{Lema:} \texttt{mapA f s} es $O( \sum\limits_{i=0}^{|s|-1} W[f\ s_i] )$ en trabajo y $O( \max\limits_{i=0}^{|s|-1} S[f\ s_i] )$ en profundidad.\\
 \textit{Dem:} Se desprende como corolario de las cotas de \texttt{tabulate f n} con $g\ i = f\ s_i$. 
 
\end{itemize}

Del primer lema anterior, resulta que $W[g\ y]\in O(W[p\ y])$ y $S[g\ y]\in O(S[p\ y])$.

Del segundo lema anterior, resulta \texttt{map\ g\ x} es $O(\sum\limits_{i=0}^{|x|-1} W[p\ x_i])$ en trabajo y $O(\max\limits_{i=0}^{|x|-1} S[p\ x_i])$ en profundidad.

Tenemos entonces por la especificacion de \texttt{flatten} que \texttt{filter\ p\ x} es $O(|x| + \sum\limits_{i=0}^{|x|-1} W[p\ x_i])$ en trabajo ya que $|g(y)| \in O(1)$;y es $O(\log |x| + \max\limits_{i=0}^{|x|-1} S[p\ x_i])$ en profundidad.


\end{document}